%
% File acl2012.tex
%
% Contact: Maggie Li (cswjli@comp.polyu.edu.hk), Michael White (mwhite@ling.osu.edu)
%%
%% Based on the style files for ACL2008 by Joakim Nivre and Noah Smith
%% and that of ACL2010 by Jing-Shin Chang and Philipp Koehn


\documentclass[11pt]{article}
\usepackage{acl2012}
\usepackage{times}
\usepackage{latexsym}
\usepackage{amsmath}
\usepackage{multirow}
\usepackage{url}
\DeclareMathOperator*{\argmax}{arg\,max}
\setlength\titlebox{6.5cm}    % Expanding the titlebox

\title{Emotional Responses to Poetry}

\author{P. Thomas Barthelemy \\
  Computer Science\\
  Stanford University \\
  {\tt bartho@stanford.edu} \\\And
  Rob Voigt \\
  East Asian Studies \\
  Stanford University \\
  {\tt robvoigt@stanford.edu} \\\And
  Jean Y. Wu \\
  Symbolic Systems  \\
  Stanford University\\
  {\tt jeaneis@stanford.edu} \\}

\date{}

\begin{document}
\maketitle
\begin{abstract}
  This document contains the instructions for preparing a camera-ready manuscript for the proceedings of ACL2012. The document itself conforms to its own specifications, and is therefore an example of what your manuscript should look like. These instructions should be used for both papers submitted for review and for final versions of accepted papers. Authors are asked to conform to all the directions reported in this document.
\end{abstract}

\section{Introduction}

\paragraph{}
\emph{Poetry is when an emotion has found its thought and the thought has found words.}
-Robert Frost

Literature in general and poetry in particular present unique challenges for natural language understanding systems. Literary scholars often articulate the manner in which the primary purpose of literature is deviance, in some sense, from the common expectations we hold of human language. Raymond Chapman describes literature as “the art that uses language,” and Viktor Shklovskij notes that in poetry in particular we consistently find “material obviously created to remove the automatism of perception.” In Shklovskij's terms, literature effects a “defamiliarization” that surprises, delights, and moves to emotion in a way normal language does not.

Beyond domain adaptation problems, divergences in structure and language use in poetic texts introduce difficulties for traditional NLP and NLU such as syntactic or semantic parsers, which often rely on some expectation of “well-formedness,” either explicitly or implicitly (such as in training data comprised of only full, “real” sentences). In our project, we are interested in discovering the features of poetic texts that correlate highly with a strong emotional response on the part of the reader. It is a fascinating aspect of poetry that it is often able to deliver a high emotional impact to a reader concisely, and in one sense this is an aspect of poetry that sets it apart from other genres of text. Therefore, our papers selected for this literature review concern work that attempts to confront the aforementioned difficulties of a computational understanding of poetry, and then work that addresses the emotional content of language.

\section{Related Work}

\newcite{kao2012computational}'s work ...


\section{Dataset}

\section{Conclusion}

\section{Future work}





\section*{Acknowledgments}

Do not number the acknowledgment section. Do not include this section when submitting your paper for review.

\bibliographystyle{acl}
\bibliography{bib}

\end{document}
